%DO NOT MESS AROUND WITH THE CODE ON THIS PAGE UNLESS YOU %REALLY KNOW WHAT YOU ARE DOING
\chapter{Literature Survey} \label{Literature Survey}
\subsection{Papers Reviewed} \label{Papers Reviewed}
A paper proposed by Patrick Olivier Kamgueu et. al focuses on the analysis of combining several metrics criteria for the implementation of RPL objective function, the new routing standard for the Internet of Things. The general problem is known as NPcomplete, we propose the use of fuzzy inference system for finding a good trade-off among the various chosen metrics. Many routing solutions tend to favour increase on network lifetime, neglecting some  other network performance aspects. In this work, they consider : the expected number of transmission needed to successfully send a packet to its final destination, to meet reliability; the latency, to minimize end-to-end delay; in addition to the remaining power draw by node, for network lifetime extension. Implementation was done on Contiki and simulations were carried out on its emulator Cooja. Obtained results show improvements compared with those from the most common implementation, namely the one that uses ETX as unique routing metric\cite{1}.
\\
\\
The paper proposed by Adeeb Saaidah et. al discusses the nature of the Low power and lossy networks (LLNs) requires having efficient protocols capable of handling the resource constraints. LLNs consist of networks that connect different type of devices which has constraints resources such as energy, memory and battery life. Using the standard routing protocols such as Open Shortest Path First (OSPF) is inefficient for LLNs due to the constraints that LLNs need. IPv6 Routing Protocol for Low-Power and Lossy Networks (RPL) was developed to accommodate these constraints. RPL is a distance vector protocol that uses the object functions (OF) to define the best tree path. Choosing a single metric for the OF found to be unable to accommodate applications requirements. In this paper, an enhanced (OF) is proposed namely; OFRRT-FUZZY relying on several metrics combined using Fuzzy Logic. In order to overcome the limitations of using a single metric, the proposed OFRRT-FUZZY considers node and link metrics. Namely, Received Signal Strength Indicator (RSSI), Remaining Energy (RE) and Throughput (TH). The proposed OFRRT-FUZZY is implemented under Cooja simulator and then results were compared with OF0, MHROF in order to find which OF provides more satisfactory results. And simulation results show that OFRRT-FUZZY outperformed OF0 and MHROF\cite{2}.
\\
\\
The paper proposed by Muneer Bani et. al describes how recently, IETF standardised a powerful and flexible Routing Protocol for Low Power and Lossy Networks (RPL). It selects the ideal routes from a source to a destination node based on certain metrics injected into the Objective Function (OF). In this study, the performance of RPL has been investigated in terms of two OFs (i.e. Minimum Rank with Hysteresis Objective Function (MRHOF) and Objective Function Zero (OF0)) in various topologies (grid, random) which makes this work distinctive. To study the RPL OFs performance, various parameters are considered Packet Delivery Ratio (PDR), Power Consumption and RX. The evaluation has been conducted based on these parameters (RX, topology) and compared for both OFs. The simulation results revealed that these parameters have a great impact on the PDR and achieved saved energy levels in the given networks. Our results have also indicated that the performance of RPL within light density networks for MRHOF can provide a better RPL behavior that OF0 could not provide\cite{3}.
\\
\\
The paper proposed by Hanane Lamaazi RPL  routing  protocol is designed  to  respond  to the requirements  of  a  large  range  of  Low-power  and  Lossy  Networks (LLNs). RPL uses an objective function (OF) to build the route toward a  destination  based  on  routing  metrics.  Considering  only  a  single metric, some network performances can be improved while others may be degraded. In  this paper, we present  a  flexible Objective  Function based on Expected Transmission Count (ETX), Consumed Energy and Forwarding Delay (OF-ECF) built on a combination of metrics using an  additive  method. The  main  goal  of  this  proposed  solution  is to balance  energy  consumption  and  minimize  the  average  delay.  To improve the reliability  of the network, a flexible routing  scheme that provides the diversity of paths  and  a higher availability is presented. Simulation  results  show  that  the  new  objective  function  OF-ECF outperforms the OF-FUZZY, and the standards OF0 and MRHOF.  In terms of network lifetime and reliability\cite{4}.
\\
\\
The paper proposed by Tayyab Mehbood demonstrates the scheme regarding Internet of Things (IOT) which is well thought-out the next generation of Internet. IOT explicitly  elaborates  the assimilation of human beings and physical systems, as they can cooperate with each other so leading towards a sort of encroachment in networking by interconnecting things  together while making use of wireless embedded systems, said to be the building blocks of IOT, that are capable to be given an IP address and thus making them part of the  global  internet. Several  essential approaches that entail in IOT and supports this innovation are being argued in this paper. 6LoWPAN (IPV6 Low Power Personal Area Networks) is a protocol used to  appropriately  and  efficiently use IPV6 addresses. Control messages of RPL routing  protocol  for  low power devices are discussed to understand the working of RPL protocol. In the end  Contiki OS based COOJA Network simulator is used to demonstrate the working of how  these routing and compression protocol works in real time simulation\cite{5}.
