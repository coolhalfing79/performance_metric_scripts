\begin{center}
\begin{huge}
\bfseries{List of Publications}
\end{huge}
\end{center}
\noindent List of Publication should be in IEEE format
\pagebreak
